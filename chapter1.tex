\chapter{Problem Restatement}
\section{Problem Background}
In the era of information explosion, news texts have become a crucial source for people to access social dynamics, public opinion, and emotional responses. Particularly against the backdrop of widespread dissemination on social media and news websites, the emotional information embedded in news language not only influences public understanding of events but also significantly shapes the evolution of public sentiment. Therefore, efficiently and accurately extracting emotional features from news texts holds substantial practical significance for government governance, corporate public opinion monitoring, and media content management.\\
\indent Against this backdrop, accurately identifying emotional information in news texts is crucial for understanding social attitudes and public tendencies. However, due to the complexity of the Chinese language, the subtlety of emotional expression, and the combination of objective statements with subjective commentary in news texts, real-world data often exhibits class imbalance in emotional expression. Simultaneously, the diverse forms of emotional expression in texts lead to issues such as homogeneity, sparsity, or weak discriminative power in features. Therefore, there is an urgent need to leverage natural language processing to construct sentiment classification models with strong generalization capabilities to identify and categorize emotions from massive news data.

\section{Specific Problem}
\indent In this project, our task is to perform ternary classification of sentiment polarity for 2,621 news sentences from 166 articles, determining whether each sentence's emotional tendency is positive, negative, or neutral. The original corpus labels are divided into four categories: no sentiment, positive sentiment, negative sentiment, and neutral sentiment. Notably, if a neutral sentiment is classified as positive or negative, it will be considered correct; if classified as no sentiment, it will be considered incorrect and randomly treated as positive or negative sentiment (reflected in the confusion matrix).\\
\indent The key to completing this task lies in performing effective feature extraction on these sentences to capture the critical emotional information of each sentence, thereby facilitating subsequent sentiment classification. After feature extraction, we will conduct sentiment classification, determining the emotional polarity based on the feature information of each sentence. To evaluate model performance, we will perform three random partitions of the training and test sets (7:3 ratio) for experiments, ultimately calculating the model's average accuracy, average precision, average recall, and average F1-Score as evaluation metrics.\\
\indent Among these, accuracy measures the proportion of correctly classified sentiments among all predictions; precision measures the proportion of sentences actually belonging to a specific sentiment polarity among all predictions classified as that polarity; recall measures the proportion of sentences correctly identified by the model among all sentences actually belonging to a specific sentiment polarity; F1-Score is the harmonic mean of precision and recall, providing a comprehensive evaluation of model performance.\\
\indent Based on these evaluation metrics, we can comprehensively assess the model's effectiveness in handling sentiment classification tasks, providing data support and basis for subsequent optimization and improvement.